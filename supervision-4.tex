\documentclass{supervision}
\usepackage{course}

\Supervision{4}

\begin{document}
  \begin{questions}
    \section*{Past exam paper questions}
    \question 2012 Paper 4 Question 3
      \begin{parts}
        \part[2] What are the main criteria to be considered in the design of a
          line drawing algorithm for a raster graphics display?
          \begin{solution}
            \begin{itemize}
              \item Make the line appear to have uniform thickness.
              \item Don't leave gaps in the line
            \end{itemize}
          \end{solution}

        \part[6] Describe an algorithm to fill a series of pixels running from
          $(x_0, y_0)$ to $(x_1, y_1)$ that meets these criteria, explaining
          why it does so. Answers should consist of more than a fragment of
          pseudo-code.
          \begin{solution}
            \begin{python}
                def draw_line(x0,y0,x1,y1):
                    # The algorithm relies upon these assumptions:
                    # - x0 < x1 (line goes left to right)
                    # - the gradient is between -1 and 1 (otherwise the line
                    #   is too steep and gaps appear)
                    #
                    # The following section of code checks these conditions
                    # and swaps the coordinates around so that they are
                    # satisfied

                    flipped = False # True if x and y have been swapped with
                                    # each other

                    if abs(y1-y0) > abs(x1-x0):
                        flipped = True
                        x0, x1, y0, y1 = y0, y1, x0, x1

                    if x0 > x1:
                        x0, x1, y0, y1 = x1, x0, y1, y0

                    delta_x = x1 - x0
                    delta_y = abs(y1 - y0)
                    error = delta_x / 2

                    y = y0
                    y_step = 1 if y0 < y1 else -1

                    # For each column draw a single pixel, if error becomes
                    # positive then we need to either increment or decrement y
                    # for the next pixel.
                    for x in range(x0, x1+1):
                        if flipped:
                            draw(y, x)
                        else:
                            draw(x, y)
                        error -= delta_y
                        if error < 0:
                            y += y_step
                            error += delta_x

                    # This algorithm ensures that the line has uniform
                    # thickness with no gaps by plotting a single pixel for
                    # each column. For gradients not between -1 and 1 it swaps
                    # them first (thus plotting a single pixel for each row
                    # instead)
            \end{python}
          \end{solution}

        \part[6] A new volumetric display stores an image as a
          three-dimensional array of volume elements or voxels. Reformulate the
          design and implementation of the line-drawing algorithm to fill a
          series of voxels running from $(x0,y0,z0)$ to $(x1, y1, z1)$.
          \begin{solution}
            \begin{python}
              def draw_line_3d(x0,y0,z0,x1,y1,z1):
                  # The algorithm relies upon these assumptions:
                  # - x0 < x1 (line goes left to right)
                  # - delta_x > delta_y and delta_x > delta_z
                  #
                  # The following section of code checks these conditions
                  # and swaps the coordinates around so that they are
                  # satisfied

                  x_flipped_with = False

                  if abs(y1-y0) > abs(x1-x0) and abs(y1-y0) > abs(z1-z0):
                      flipped = "y"
                      x0, x1, y0, y1 = y0, y1, x0, x1

                  if abs(z1-z0) > abs(x1-x0) and abs(z1-z0) > abs(y1-y0):
                      flipped = "z"
                      x0, x1, z0, z1 = z0, z1, x0, x1

                  if x0 > x1:
                      x0, y0, z0, x1, y1, z1 = x1, y1, z1, x0, y0, z0

                  delta_x = x1 - x0
                  delta_y = abs(y1 - y0)
                  delta_z = abs(z1 - z0)
                  error_y = error_z = delta_x / 2

                  y = y0
                  z = z0
                  y_step = 1 if y0 < y1 else -1
                  z_step = 1 if z0 < z1 else -1

                  # For each column draw a single pixel, if error becomes
                  # positive then we need to either increment or decrement y/z
                  # for the next pixel.
                  for x in range(x0, x1+1):
                      if flipped == "y":
                          draw(y, x, z)
                      elif flipped == "z":
                          draw(z, y, x)
                      else:
                          draw(x, y, z)
                      error_y -= delta_y
                      error_z -= delta_z

                      if error_y < 0:
                          y += y_step
                          error_y += delta_x


                      if error_z < 0:
                          z += z_step
                          error_z += delta_x
            \end{python}
          \end{solution}

        \part[6] How would this line-drawing algorithm be used to draw Bézier
          curves in three dimensions?
          \begin{solution}
            \begin{enumerate}
              \item Determine whether the line from $p_0$ to $p_3$ is an
                adequate aproximation by checking whether it is within a
                certain tolerance of the Bézier curve. If it is then draw the
                line and stop, otherwise continue.
              \item Subdivide the curve into two bezier curves and return to
                step 1 for each.
            \end{enumerate}
          \end{solution}

      \end{parts}
    \question 2011 Paper 4 Question 3
      You are writing code for a new graphics card that is software
      programmable, rather than having the algorithms embedded in hardware. You
      want to write a fast triangle-drawing program to test the card.

      \begin{parts}
        \part[13] Provide pseudocode, or similar, that draws a triangle with a
          constant colour. Assume that the inputs are the colour of the
          triangle and three two-dimensional points, representing the three
          vertices of the triangle. Further, assume that all three vertices lie
          on the visible screen and that no anti-aliasing is required. You may
          assume that there is a function to set a pixel, $(x, y)$, to a
          particular colour, $c$, e.g. ${setpixel}(x,y,c)$, but you must
          provide pseudocode for any other functions that you require. Your
          answer should be sufficiently detailed that it could be transferred
          directly into a language such as Java but your answer does not,
          itself, have to be syntactically correct.
          \begin{solution}
            \begin{python}
              def draw_triangle(x0, y0, x1, y1, x2, y2, c):
                if y2 < y1:
                    x1, y1, x2, y2 = x2, y2, x1, y1

                if y1 < y0:
                    x0, y0, x1, y1 = x1, y1, x0, y0

                if y2 < y1:
                    x1, y1, x2, y2 = x2, y2, x1, y1

                # coordinates are now sorted by y

                gradient_1 = (x1 - x0) / (y1 - y0)
                gradient_2 = (x2 - x0) / (y2 - y0)

                for row in range(y0, y1):
                    intersection_1 = x0 + (row - y0) * gradient_1
                    intersection_2 = x0 + (row - y0) * gradient_2

                    if intersection_2 < intersection_1:
                        intersection_1, intersection_2 = (intersection_2,
                            intersection_1)

                    for col in range(intersection_1, intersection_2+1):
                        setpixel(col, row, c)

                gradient_1 = (x1 - x2) / (y1 - y2)

                for row in range(y1, y2):
                    intersection_1 = x0 + (row - y2) * gradient_1
                    intersection_2 = x0 + (row - y0) * gradient_2

                    if intersection_2 < intersection_1:
                        intersection_1, intersection_2 = (intersection_2,
                            intersection_1)

                    for col in range(intersection_1, intersection_2+1):
                        setpixel(col, row, c)
            \end{python}
          \end{solution}

        \part[4] Outline the extra steps required to draw a triangle specified
          by three-dimensional points in world space, where the triangle may
          extend beyond the edges of the screen after conversion to screen
          space.
          \begin{solution}
            \begin{itemize}
              \item Project the coordinates onto the 2D screen space.
              \item clip to the edge of the screen (splitting into smaller
                triangles)
            \end{itemize}
          \end{solution}

        \part[3] Outline the steps required to calculate the triangle’s colour,
          assuming diffuse shading, with multiple point lights, but still
          producing a single colour for the whole triangle.
          \begin{solution}
            For each pixel that is part of the triangle calculate the diffuse
            shading using the normal to the surface of the triangle and the
            direction to each of the light sources. Combine these to get the
            overall shading for that pixel.
          \end{solution}
      \end{parts}

    \section*{Practical Programming}
    \question Implement one curve drawing algorithm in 2D. Note that your
      algorithm should use a line drawing algorithm to draw curve segments.
      \begin{solution}
        \pythonfile{3-curve-drawing/draw_curve.py}
      \end{solution}
  \end{questions}
\end{document}
