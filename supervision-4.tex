\documentclass{supervision}
\usepackage{course}

\Supervision{4}

\begin{document}
  \begin{questions}
    \section*{Past exam paper questions}
    \question 2012 Paper 4 Question 3
      \begin{parts}
        \part[2] What are the main criteria to be considered in the design of a
          line drawing algorithm for a raster graphics display?

        \part[6] Describe an algorithm to fill a series of pixels running from
          $(x0,y0)$ to $(x1,y1)$ that meets these criteria, explaining why it
          does so. Answers should consist of more than a fragment of
          pseudo-code.

        \part[6] A new volumetric display stores an image as a
          three-dimensional array of volume elements or voxels. Reformulate the
          design and implementation of the line-drawing algorithm to fill a
          series of voxels running from $(x0,y0,z0)$ to $(x1, y1, z1)$.

        \part[6] How would this line-drawing algorithm be used to draw Bézier
          curves in three dimensions?

      \end{parts}
    \question 2011 Paper 4 Question 3
      You are writing code for a new graphics card that is software
      programmable, rather than having the algorithms embedded in hardware. You
      want to write a fast triangle-drawing program to test the card.

      \begin{parts}
        \part[13] Provide pseudocode, or similar, that draws a triangle with a
          constant colour. Assume that the inputs are the colour of the
          triangle and three two-dimensional points, representing the three
          vertices of the triangle. Further, assume that all three vertices lie
          on the visible screen and that no anti-aliasing is required. You may
          assume that there is a function to set a pixel, $(x, y)$, to a
          particular colour, $c$, e.g. ${setpixel}(x,y,c)$, but you must
          provide pseudocode for any other functions that you require. Your
          answer should be sufficiently detailed that it could be transferred
          directly into a language such as Java but your answer does not,
          itself, have to be syntactically correct.

        \part[4] Outline the extra steps required to draw a triangle specified
          by three-dimensional points in world space, where the triangle may
          extend beyond the edges of the screen after conversion to screen
          space.

        \part[3] Outline the steps required to calculate the triangle’s colour,
          assuming diffuse shading, with multiple point lights, but still
          producing a single colour for the whole triangle.

      \end{parts}

    \section*{Practical Programming}
    \question Implement one curve drawing algorithm in 2D. Note that your
      algorithm should use a line drawing algorithm to draw curve segments.
  \end{questions}
\end{document}
