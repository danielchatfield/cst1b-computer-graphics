\documentclass{supervision}
\usepackage{course}
\usepackage{fancyvrb}

\Supervision{3}
\Topics{Colour, displays and image processing}

\begin{document}

\section*{From Supervision 2}
\begin{questions}
    \question
    Derive the conditions necessary for two Bézier curves to join with:
    \begin{parts}
        \part{just C0-continuity}
        \part{C1-continuity}
        \part{C2-continuity}
    \end{parts}

    \question
    What would be difficult about getting three Bézier curves to join in sequence with C2-continuity at the two joins?

    \question
    For a cylinder of radius 2, with endpoints $(1,2,3)$ and $(2,4,5)$, show how to calculate:
    \begin{parts}
        \part{an axis-aligned bound box}
        \part{a bounding sphere}
    \end{parts}

    \question
    Break down the following (2D!) lines into a BSP-tree, splitting them if necessary:
    \begin{itemize}
        \item{$(0,0)-(2,2)$}
        \item{$(3,4)-(1,3)$}
        \item{$(1,0)-(-3,1)$}
        \item{$(0,3)-(3,3)$}
        \item{$(2,0)-(2,1)$}
    \end{itemize}

    \question
    \begin{parts}
        \part{Compare the two methods of doing 3D clipping in terms of efficiency}
        \part{How would using bounding volumes improve efficiency of these methods?}
    \end{parts}

    \question
    Describe a complete algorithm to do 3D polygon scan conversion, including details of clipping, projection, and the underlying 2D polygon scan conversion algorithm.

    \question
    Describe how you would form a good approximation to a cylinder from Bézier patches. Draw the patches and their control points and give the coordinates of the control points.

    \question
    Given the following sixteen points, calculate the first eight of the next patch joining it as $t$ increases so that the join has continuity $C1$. Here the points are listed with $s=0$, $t=0$ on the bottom left, with $s$ increasing upwards and $t$ increasing to the right:
    \begin{Verbatim}[fontsize=\scriptsize]
(-0.2, 3.4,  0.3)    (1.0, 3.1, 0.2)    (2.0, 2.6, -0.2)    (3.1, 2.8,  0.2)
( 0.0, 1.2,  0.4)    (1.2, 2.0, 1.2)    (1.4, 1.9, -0.2)    (2.7, 1.8,  0.2)
( 0.2, 1.0, -0.2)    (1.1, 0.8, 0.5)    (1.4, 1.0,  0.0)    (3.1, 1.1, -0.2)
( 0.0, 0.0,  0.0)    (1.0, 0.0, 0.5)    (2.0, 0.2,  0.4)    (2.7, 0.0, -0.2)
    \end{Verbatim}

\end{questions}

\section*{\Topics}
\subsection*{Warmup Questions}
\begin{questions}
    \question
    Compare and contrast the use of LCDs and electrophoretic displays for screens in portable devices.

    \question
    Compare the rendering in some different pieces of printed material. Use a magnifying glass to explore the resolution, colours and patterns used.

\end{questions}

\subsection*{Longer Questions}
\begin{questions}
    \question
    Explain the use of each of the following colour spaces:
    \begin{parts}
        \part{$RGB$}
        \part{$XYZ$}
        \part{$HLS$}
        \part{$Luv$}
    \end{parts}
    \question
    Explain the difference between additive colour (RGB) and subtractive colour (CMY). Where is each used and why is it used there?

    \question
    Compare the two methods of \emph{Error Diffusion} described in the notes, with the aid of a sample image.

\end{questions}



\end{document}
