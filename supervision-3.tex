\documentclass{supervision}
\usepackage{course}
\usepackage{fancyvrb}
\usepackage{qtree}

\Supervision{3}
\Topics{Colour, displays and image processing}

\begin{document}

\section*{From Supervision 2}
\begin{questions}
  \question Derive the conditions necessary for two Bézier curves to join with:
    \begin{solution}
      Consider two bezier curves $p(t)$ and $q(t)$, defined by control points
      $(p_0,p_1,..., p_n)$ and $(q_0, q_1,..., q_n)$ respectively.
    \end{solution}
  \begin{parts}
    \part just C0-continuity
      \begin{solution}
        $C0$ continuity can be achieved by setting $p(1) = q(0)$ giving:
        \begin{center}
          $q_0 = p_n$
        \end{center}
      \end{solution}
    \part C1-continuity
      \begin{solution}
        $C1$ continuity can be achieved with the additional constraint that
        $p'(1) = q'(0)$ giving:
        \begin{center}
          $q_1 - q_0 = p_n - P_{n-1}$
        \end{center}

        Which when combined with the above constraint gives:
        \begin{center}
          $q_1 = 2p_n - p_{n-1}$
        \end{center}
      \end{solution}
    \part C2-continuity
      \begin{solution}
        For $C2$ continuity we must also have that $p''(1) = q''(0)$, giving:
        \begin{center}
          $q_2 - 2q_1 + q_0 = p_n - 2P_{n-1} + p_{n-2}$
        \end{center}

        Which when combined with the above constraint gives:
        \begin{center}
          $q_2 = p_{n-2} + 4(p_n - p_{n-1})$
        \end{center}
      \end{solution}
  \end{parts}

  \question What would be difficult about getting three Bézier curves to join
    in sequence with C2-continuity at the two joins?
    \begin{solution}

    \end{solution}

  \question
    For a cylinder of radius $2$, with endpoints $(1,2,3)$ and $(2,4,5)$, show
    how to calculate:
    \begin{parts}
      \part an axis-aligned bounding box
        \begin{solution}
          Let $m$, $n$, and $o$ represent the $x$, $y$, and $z$ axes
          respectively.

          Let $A$ and $B$ be the endpoints such that $A_M <= B_M$.

          Let $
          \delta =
          \frac{\sqrt{(A_n - B_n)^2 + (A_o - B_o)^2}}
               {(A_m - B_m)^2 + (A_n - B_n)^2 + (A_o - B_o)^2}
          $

          $m_{min} = A_m - \delta \times r$ \\
          $m_{max} = B_m + \delta \times r$


        \end{solution}

      \part a bounding sphere
        \begin{solution}
          % TODO Image here
          The bounding sphere has midpoint that is half way between the two
          endpoints:
          ${mid} = (1.5, 3, 4)$

          The radius is the distance from the midpoint to a point on the edge
          of the cylinder.

          One such point is $(2, 4 + \sqrt{2}, 5 - \sqrt{2})$.

          Giving: $r = \sqrt{(0.5)^2 + (1+\sqrt{2})^2 + (1-\sqrt{2})^2} = 2.5$
        \end{solution}
    \end{parts}

  \question Break down the following (2D!) lines into a BSP-tree, splitting
    them if necessary:
    \begin{description}
      \item[a] $(0,0)-( 2,2)$
      \item[b] $(3,4)-( 1,3)$
      \item[c] $(1,0)-(-3,1)$
      \item[d] $(0,3)-( 3,3)$
      \item[e] $(2,0)-( 2,1)$
    \end{description}
    \begin{solution}
      % TODO image here
      \Tree[.a
        [.d b   c_1 ]
        [.e c_2     ]
      ]
    \end{solution}

  \question
    \begin{parts}
      \part Compare the two methods of doing 3D clipping in terms of
        efficiency
        \begin{solution}
          \emph{Not sure which is more efficient - both require clipping
          against 6 axes.}
        \end{solution}
      \part How would using bounding volumes improve efficiency of these
        methods?
        \begin{solution}
          Computing and storing bounding volumes for each object in the scene
          makes it much easier to quickly determine that two objects don't
          intersect.
        \end{solution}
    \end{parts}

  \question Describe a complete algorithm to do 3D polygon scan conversion,
    including details of clipping, projection, and the underlying 2D polygon
    scan conversion algorithm.
    \begin{solution}
      % TODO
    \end{solution}

  \question Describe how you would form a good approximation to a cylinder
    from Bézier patches. Draw the patches and their control points and give
    the coordinates of the control points.
    \begin{solution}
      % TODO
    \end{solution}

  \question Given the following sixteen points, calculate the first eight of
    the next patch joining it as $t$ increases so that the join has continuity
    $C1$. Here the points are listed with $s=0$, $t=0$ on the bottom left,
    with $s$ increasing upwards and $t$ increasing to the right:
    \begin{Verbatim}[fontsize=\scriptsize]
(-0.2, 3.4,  0.3)    (1.0, 3.1, 0.2)    (2.0, 2.6, -0.2)    (3.1, 2.8,  0.2)
( 0.0, 1.2,  0.4)    (1.2, 2.0, 1.2)    (1.4, 1.9, -0.2)    (2.7, 1.8,  0.2)
( 0.2, 1.0, -0.2)    (1.1, 0.8, 0.5)    (1.4, 1.0,  0.0)    (3.1, 1.1, -0.2)
( 0.0, 0.0,  0.0)    (1.0, 0.0, 0.5)    (2.0, 0.2,  0.4)    (2.7, 0.0, -0.2)
    \end{Verbatim}
    \begin{solution}
      % TODO
    \end{solution}
\end{questions}

\section*{\Topics}
\subsection*{Warmup Questions}
\begin{questions}
  \question Compare and contrast the use of LCDs and electrophoretic displays
    for screens in portable devices.

  \question Compare the rendering in some different pieces of printed
    material. Use a magnifying glass to explore the resolution, colours and
    patterns used.

\end{questions}

\subsection*{Longer Questions}
\begin{questions}
  \question Explain the use of each of the following colour spaces:
    \begin{parts}
      \part{$RGB$}
      \part{$XYZ$}
      \part{$HLS$}
      \part{$Luv$}
    \end{parts}
  \question Explain the difference between additive colour (${RGB}$) and
    subtractive colour (${CMY}$). Where is each used and why is it used there?

  \question Compare the two methods of \emph{Error Diffusion} described in the
    notes, with the aid of a sample image.

\end{questions}
\end{document}
